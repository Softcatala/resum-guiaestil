\documentclass[9pt]{cheatsheet}

\cheatsheettitle{Guia d'estil de Softcatalà (resum)}

\begin{document}

\begin{multicols*}{3}

\cheatsheetsectiontitle{Aspectes lingüístics}

\cheatsheetsection{Veu activa i veu passiva}

Sempre intentarem convertir les frases a la forma activa:

\underline {Text original} Your signature \textbf{is not displayed}, but \textbf{is added} to the message window when the message \textbf{is sent}.

\underline {No} La vostra signatura \textbf{no és visualitzada}, però \textbf{és afegida} a la finestra del missatge quan \textbf{és enviat}.

\underline {Sí} La vostra signatura \textbf{no es visualitza}, però \textbf{s'afegeix} a la finestra del missatge quan \textbf{s'envia}


\cheatsheetsection{Formes verbals}


Quan és l’usuari qui s'adreça a l'ordinador, s'utilitzarà sempre l’imperatiu en segona persona del singular (que correspon al tractament de \emph{tu}). Això ho trobarem sovint en els menús i en alguns quadres de diàleg, especialment els d’opcions de configuració (sempre que no siguin quadres de diàleg en què l’ordinador ens dóna alguna informació o ens pregunta alguna cosa).

D’aquesta manera, no utilitzarem \emph{Editar}, sinó \emph{Edita}, ni \emph{Obrir}, sinó \emph{Obre}, per a expressar la força imperativa que, en l’àmbit TIC anglòfon, es dóna a aquesta mena d’enunciats.



\cheatsheetsectiontitle{Aspectes de localització}

\cheatsheetsection{Decimals i milers}

L’ús del punt i la coma com a separadors de decimals i milers canvia segon els països. Per exemple, als Estats Units els milers se separen amb una coma i els decimals amb un punt, mentre que en català, segons les convencions dels principals territoris on es parla, es fa a l’inrevés:

\underline {Anglès} 1,234,567.89 (als Estats Units)

\underline {Català} 1.234.567,89


\cheatsheetsection{Data i hora}

En anglès, les hores s’expressen amb el format de 12 hores (AM o PM); en català, ho farem amb un format de 24 hores i separant l’hora dels minuts i els minuts dels segons per mitjà de dos punts:

Un quart de sis de la tarda, en anglès: \textbf{5:15:00 PM}.

La mateixa hora, en català: \textbf{17:15:00}.

El format en què s’expressen les dates depèn de cada país. Als Estats Units, les dates s’expressen com mes/dia/any i en alguns països, com ara Japó, s’expressen com any/mes/dia. En català, utilitzem sempre dia/mes/any (l’any s’escriu sense punt separador de milers):

\underline {Anglès} September 11th, 2000.

\underline {Català}	11 de setembre de 2000.

\cheatsheetsection{Localització de documentació}

\cheatsheetsubsection{Títols}

En anglès, és habitual emprar el gerundi en els títols de seccions de manuals o noms de documents. En català, cal defugir el gerundi i utilitzar preferentment construccions nominals o, si això no és possible, infinitius. Per exemple, traduirem el títol \emph{Adding Graphics to the Gallery} per \emph{Addició de gràfics a la galeria} o, alternativament, \emph{Afegir gràfics a la galeria}.

\cheatsheetfooter{(c)2020 Softcatalà}{https://www.softcatala.org}
\end{multicols*}
\end{document}
